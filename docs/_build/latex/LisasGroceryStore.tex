% Generated by Sphinx.
\def\sphinxdocclass{report}
\documentclass[letterpaper,10pt,english]{sphinxmanual}
\usepackage[utf8]{inputenc}
\DeclareUnicodeCharacter{00A0}{\nobreakspace}
\usepackage{cmap}
\usepackage[T1]{fontenc}
\usepackage{babel}
\usepackage{times}
\usepackage[Bjarne]{fncychap}
\usepackage{longtable}
\usepackage{sphinx}
\usepackage{multirow}

\addto\captionsenglish{\renewcommand{\figurename}{Fig. }}
\addto\captionsenglish{\renewcommand{\tablename}{Table }}
\floatname{literal-block}{Listing }



\title{Lisa's Grocery Store Documentation}
\date{July 28, 2015}
\release{1.0}
\author{Lisa Kugler}
\newcommand{\sphinxlogo}{}
\renewcommand{\releasename}{Release}
\makeindex

\makeatletter
\def\PYG@reset{\let\PYG@it=\relax \let\PYG@bf=\relax%
    \let\PYG@ul=\relax \let\PYG@tc=\relax%
    \let\PYG@bc=\relax \let\PYG@ff=\relax}
\def\PYG@tok#1{\csname PYG@tok@#1\endcsname}
\def\PYG@toks#1+{\ifx\relax#1\empty\else%
    \PYG@tok{#1}\expandafter\PYG@toks\fi}
\def\PYG@do#1{\PYG@bc{\PYG@tc{\PYG@ul{%
    \PYG@it{\PYG@bf{\PYG@ff{#1}}}}}}}
\def\PYG#1#2{\PYG@reset\PYG@toks#1+\relax+\PYG@do{#2}}

\expandafter\def\csname PYG@tok@gd\endcsname{\def\PYG@tc##1{\textcolor[rgb]{0.63,0.00,0.00}{##1}}}
\expandafter\def\csname PYG@tok@gu\endcsname{\let\PYG@bf=\textbf\def\PYG@tc##1{\textcolor[rgb]{0.50,0.00,0.50}{##1}}}
\expandafter\def\csname PYG@tok@gt\endcsname{\def\PYG@tc##1{\textcolor[rgb]{0.00,0.27,0.87}{##1}}}
\expandafter\def\csname PYG@tok@gs\endcsname{\let\PYG@bf=\textbf}
\expandafter\def\csname PYG@tok@gr\endcsname{\def\PYG@tc##1{\textcolor[rgb]{1.00,0.00,0.00}{##1}}}
\expandafter\def\csname PYG@tok@cm\endcsname{\let\PYG@it=\textit\def\PYG@tc##1{\textcolor[rgb]{0.25,0.50,0.56}{##1}}}
\expandafter\def\csname PYG@tok@vg\endcsname{\def\PYG@tc##1{\textcolor[rgb]{0.73,0.38,0.84}{##1}}}
\expandafter\def\csname PYG@tok@m\endcsname{\def\PYG@tc##1{\textcolor[rgb]{0.13,0.50,0.31}{##1}}}
\expandafter\def\csname PYG@tok@mh\endcsname{\def\PYG@tc##1{\textcolor[rgb]{0.13,0.50,0.31}{##1}}}
\expandafter\def\csname PYG@tok@cs\endcsname{\def\PYG@tc##1{\textcolor[rgb]{0.25,0.50,0.56}{##1}}\def\PYG@bc##1{\setlength{\fboxsep}{0pt}\colorbox[rgb]{1.00,0.94,0.94}{\strut ##1}}}
\expandafter\def\csname PYG@tok@ge\endcsname{\let\PYG@it=\textit}
\expandafter\def\csname PYG@tok@vc\endcsname{\def\PYG@tc##1{\textcolor[rgb]{0.73,0.38,0.84}{##1}}}
\expandafter\def\csname PYG@tok@il\endcsname{\def\PYG@tc##1{\textcolor[rgb]{0.13,0.50,0.31}{##1}}}
\expandafter\def\csname PYG@tok@go\endcsname{\def\PYG@tc##1{\textcolor[rgb]{0.20,0.20,0.20}{##1}}}
\expandafter\def\csname PYG@tok@cp\endcsname{\def\PYG@tc##1{\textcolor[rgb]{0.00,0.44,0.13}{##1}}}
\expandafter\def\csname PYG@tok@gi\endcsname{\def\PYG@tc##1{\textcolor[rgb]{0.00,0.63,0.00}{##1}}}
\expandafter\def\csname PYG@tok@gh\endcsname{\let\PYG@bf=\textbf\def\PYG@tc##1{\textcolor[rgb]{0.00,0.00,0.50}{##1}}}
\expandafter\def\csname PYG@tok@ni\endcsname{\let\PYG@bf=\textbf\def\PYG@tc##1{\textcolor[rgb]{0.84,0.33,0.22}{##1}}}
\expandafter\def\csname PYG@tok@nl\endcsname{\let\PYG@bf=\textbf\def\PYG@tc##1{\textcolor[rgb]{0.00,0.13,0.44}{##1}}}
\expandafter\def\csname PYG@tok@nn\endcsname{\let\PYG@bf=\textbf\def\PYG@tc##1{\textcolor[rgb]{0.05,0.52,0.71}{##1}}}
\expandafter\def\csname PYG@tok@no\endcsname{\def\PYG@tc##1{\textcolor[rgb]{0.38,0.68,0.84}{##1}}}
\expandafter\def\csname PYG@tok@na\endcsname{\def\PYG@tc##1{\textcolor[rgb]{0.25,0.44,0.63}{##1}}}
\expandafter\def\csname PYG@tok@nb\endcsname{\def\PYG@tc##1{\textcolor[rgb]{0.00,0.44,0.13}{##1}}}
\expandafter\def\csname PYG@tok@nc\endcsname{\let\PYG@bf=\textbf\def\PYG@tc##1{\textcolor[rgb]{0.05,0.52,0.71}{##1}}}
\expandafter\def\csname PYG@tok@nd\endcsname{\let\PYG@bf=\textbf\def\PYG@tc##1{\textcolor[rgb]{0.33,0.33,0.33}{##1}}}
\expandafter\def\csname PYG@tok@ne\endcsname{\def\PYG@tc##1{\textcolor[rgb]{0.00,0.44,0.13}{##1}}}
\expandafter\def\csname PYG@tok@nf\endcsname{\def\PYG@tc##1{\textcolor[rgb]{0.02,0.16,0.49}{##1}}}
\expandafter\def\csname PYG@tok@si\endcsname{\let\PYG@it=\textit\def\PYG@tc##1{\textcolor[rgb]{0.44,0.63,0.82}{##1}}}
\expandafter\def\csname PYG@tok@s2\endcsname{\def\PYG@tc##1{\textcolor[rgb]{0.25,0.44,0.63}{##1}}}
\expandafter\def\csname PYG@tok@vi\endcsname{\def\PYG@tc##1{\textcolor[rgb]{0.73,0.38,0.84}{##1}}}
\expandafter\def\csname PYG@tok@nt\endcsname{\let\PYG@bf=\textbf\def\PYG@tc##1{\textcolor[rgb]{0.02,0.16,0.45}{##1}}}
\expandafter\def\csname PYG@tok@nv\endcsname{\def\PYG@tc##1{\textcolor[rgb]{0.73,0.38,0.84}{##1}}}
\expandafter\def\csname PYG@tok@s1\endcsname{\def\PYG@tc##1{\textcolor[rgb]{0.25,0.44,0.63}{##1}}}
\expandafter\def\csname PYG@tok@gp\endcsname{\let\PYG@bf=\textbf\def\PYG@tc##1{\textcolor[rgb]{0.78,0.36,0.04}{##1}}}
\expandafter\def\csname PYG@tok@sh\endcsname{\def\PYG@tc##1{\textcolor[rgb]{0.25,0.44,0.63}{##1}}}
\expandafter\def\csname PYG@tok@ow\endcsname{\let\PYG@bf=\textbf\def\PYG@tc##1{\textcolor[rgb]{0.00,0.44,0.13}{##1}}}
\expandafter\def\csname PYG@tok@sx\endcsname{\def\PYG@tc##1{\textcolor[rgb]{0.78,0.36,0.04}{##1}}}
\expandafter\def\csname PYG@tok@bp\endcsname{\def\PYG@tc##1{\textcolor[rgb]{0.00,0.44,0.13}{##1}}}
\expandafter\def\csname PYG@tok@c1\endcsname{\let\PYG@it=\textit\def\PYG@tc##1{\textcolor[rgb]{0.25,0.50,0.56}{##1}}}
\expandafter\def\csname PYG@tok@kc\endcsname{\let\PYG@bf=\textbf\def\PYG@tc##1{\textcolor[rgb]{0.00,0.44,0.13}{##1}}}
\expandafter\def\csname PYG@tok@c\endcsname{\let\PYG@it=\textit\def\PYG@tc##1{\textcolor[rgb]{0.25,0.50,0.56}{##1}}}
\expandafter\def\csname PYG@tok@mf\endcsname{\def\PYG@tc##1{\textcolor[rgb]{0.13,0.50,0.31}{##1}}}
\expandafter\def\csname PYG@tok@err\endcsname{\def\PYG@bc##1{\setlength{\fboxsep}{0pt}\fcolorbox[rgb]{1.00,0.00,0.00}{1,1,1}{\strut ##1}}}
\expandafter\def\csname PYG@tok@mb\endcsname{\def\PYG@tc##1{\textcolor[rgb]{0.13,0.50,0.31}{##1}}}
\expandafter\def\csname PYG@tok@ss\endcsname{\def\PYG@tc##1{\textcolor[rgb]{0.32,0.47,0.09}{##1}}}
\expandafter\def\csname PYG@tok@sr\endcsname{\def\PYG@tc##1{\textcolor[rgb]{0.14,0.33,0.53}{##1}}}
\expandafter\def\csname PYG@tok@mo\endcsname{\def\PYG@tc##1{\textcolor[rgb]{0.13,0.50,0.31}{##1}}}
\expandafter\def\csname PYG@tok@kd\endcsname{\let\PYG@bf=\textbf\def\PYG@tc##1{\textcolor[rgb]{0.00,0.44,0.13}{##1}}}
\expandafter\def\csname PYG@tok@mi\endcsname{\def\PYG@tc##1{\textcolor[rgb]{0.13,0.50,0.31}{##1}}}
\expandafter\def\csname PYG@tok@kn\endcsname{\let\PYG@bf=\textbf\def\PYG@tc##1{\textcolor[rgb]{0.00,0.44,0.13}{##1}}}
\expandafter\def\csname PYG@tok@o\endcsname{\def\PYG@tc##1{\textcolor[rgb]{0.40,0.40,0.40}{##1}}}
\expandafter\def\csname PYG@tok@kr\endcsname{\let\PYG@bf=\textbf\def\PYG@tc##1{\textcolor[rgb]{0.00,0.44,0.13}{##1}}}
\expandafter\def\csname PYG@tok@s\endcsname{\def\PYG@tc##1{\textcolor[rgb]{0.25,0.44,0.63}{##1}}}
\expandafter\def\csname PYG@tok@kp\endcsname{\def\PYG@tc##1{\textcolor[rgb]{0.00,0.44,0.13}{##1}}}
\expandafter\def\csname PYG@tok@w\endcsname{\def\PYG@tc##1{\textcolor[rgb]{0.73,0.73,0.73}{##1}}}
\expandafter\def\csname PYG@tok@kt\endcsname{\def\PYG@tc##1{\textcolor[rgb]{0.56,0.13,0.00}{##1}}}
\expandafter\def\csname PYG@tok@sc\endcsname{\def\PYG@tc##1{\textcolor[rgb]{0.25,0.44,0.63}{##1}}}
\expandafter\def\csname PYG@tok@sb\endcsname{\def\PYG@tc##1{\textcolor[rgb]{0.25,0.44,0.63}{##1}}}
\expandafter\def\csname PYG@tok@k\endcsname{\let\PYG@bf=\textbf\def\PYG@tc##1{\textcolor[rgb]{0.00,0.44,0.13}{##1}}}
\expandafter\def\csname PYG@tok@se\endcsname{\let\PYG@bf=\textbf\def\PYG@tc##1{\textcolor[rgb]{0.25,0.44,0.63}{##1}}}
\expandafter\def\csname PYG@tok@sd\endcsname{\let\PYG@it=\textit\def\PYG@tc##1{\textcolor[rgb]{0.25,0.44,0.63}{##1}}}

\def\PYGZbs{\char`\\}
\def\PYGZus{\char`\_}
\def\PYGZob{\char`\{}
\def\PYGZcb{\char`\}}
\def\PYGZca{\char`\^}
\def\PYGZam{\char`\&}
\def\PYGZlt{\char`\<}
\def\PYGZgt{\char`\>}
\def\PYGZsh{\char`\#}
\def\PYGZpc{\char`\%}
\def\PYGZdl{\char`\$}
\def\PYGZhy{\char`\-}
\def\PYGZsq{\char`\'}
\def\PYGZdq{\char`\"}
\def\PYGZti{\char`\~}
% for compatibility with earlier versions
\def\PYGZat{@}
\def\PYGZlb{[}
\def\PYGZrb{]}
\makeatother

\renewcommand\PYGZsq{\textquotesingle}

\begin{document}

\maketitle
\tableofcontents
\phantomsection\label{index::doc}


Lisa's Grocery Store is an implementation of the Item Catalog project specified in the Udacity Full
Stack Web Developer Nanodegree.


\strong{See also:}


\href{https://github.com/LisaJK/grocerystore/blob/master/README.txt}{https://github.com/LisaJK/grocerystore/blob/master/README.txt}




\strong{See also:}


\href{https://www.udacity.com/course/full-stack-web-developer-nanodegree--nd004}{https://www.udacity.com/course/full-stack-web-developer-nanodegree--nd004}




\chapter{General Description}
\label{index:general-description}\label{index:welcome-to-lisa-s-grocery-store-s-documentation}
``Lisa's Grocery Store'' is a web application that provides a list of products
within a variety of product categories.
It integrates user registration and authentication via Google or Facebook.
Authenticated users have the ability to post, edit, or delete their own products.

\textbf{Additional Functionality:}
\begin{itemize}
\item {} \begin{description}
\item[{API Endpoints: Apart from the required JSON endpoints, the app has also an}] \leavevmode
implementation of XML and Atom endpoints.

\end{description}

\item {} \begin{description}
\item[{CRUD: Read:    A product image field is added which can be read from the}] \leavevmode
database and displayed on the page.

\end{description}

\item {} \begin{description}
\item[{CRUD: Create:  A product image field can be included when a new product is}] \leavevmode
created in the database.

\end{description}

\item {} \begin{description}
\item[{CRUD: Update:  For already existing products, product images can be added,}] \leavevmode
changed and deleted.

\end{description}

\item {} \begin{description}
\item[{CRUD: Delete:  The function is implemented using POST requests and nonces to}] \leavevmode
prevent cross-site request forgeries (CSRF).

\end{description}

\item {} 
Comments:      Comments are (hopefully ;-)) thorough and concise.

\end{itemize}


\chapter{Getting Started}
\label{index:getting-started}\begin{enumerate}
\item {} 
Install Vagrant and Virtual Box as described in the course materials of the
``Full Stack Foundations''.

\item {} 
Clone the repository from GitHub
\$ git clone \href{https://github.com/LisaJK/grocerystore.git}{https://github.com/LisaJK/grocerystore.git}

\item {} 
Launch the Vagrant VM.

\item {} 
Move to the directory ``/vagrant/catalog''

\item {} 
Create and populate the database within the VM by typing
``load\_grocery\_store.py'' in the console.

\item {} 
Run the application within the VM by typing ``python application.py'' in the
console.

\item {} 
Access the application by visiting ``\href{http://localhost:8000}{http://localhost:8000}'' on your browser.

\end{enumerate}

\begin{notice}{note}{Note:}
The app was developed and tested with Chrome Version 43.0.2357.134 m.
It was also tested with Firefox 39.0.
In IE you might have to add \href{https://accounts.google.com}{https://accounts.google.com} to your trusted sites
to enable Google Login (Facebook resp.).
\end{notice}


\chapter{Project Folders and Files}
\label{index:project-folders-and-files}\begin{enumerate}
\item {} \begin{description}
\item[{README.txt:}] \leavevmode
File containing all necessary information how to download and run the project.

\end{description}

\item {} \begin{description}
\item[{application.py:}] \leavevmode
This python file contains the server-side implementation of the app.

\end{description}

\item {} \begin{description}
\item[{database\_setup.py:}] \leavevmode
This python file contains the setup of the database ``grocery\_store.db''.

\end{description}

\item {} \begin{description}
\item[{load\_grocery\_store.py:}] \leavevmode
Run this python file to add some initial users, product categories and
products to the database ``grocery\_store\_db''.

\end{description}

\item {} \begin{description}
\item[{templates folder:}] \leavevmode
This folder contains all Flask templates of the app.

\end{description}
\begin{itemize}
\item {} 
category.html

\item {} 
deleteCategory.html

\item {} 
deleteProduct.html

\item {} 
editCategory.html

\item {} 
editProduct.html

\item {} 
layout.html

\item {} 
login.html

\item {} 
newCategory.html

\item {} 
newProduct.html

\item {} 
product.html

\item {} 
products.html

\item {} 
showCategory.html

\item {} 
showGroceryStore.html

\item {} 
showProduct.html

\end{itemize}

\end{enumerate}

\begin{DUlineblock}{0em}
\item[] 
\end{DUlineblock}
\begin{enumerate}
\setcounter{enumi}{5}
\item {} 
static folder:
\begin{itemize}
\item {} 
styles.css: CSS file containing the styles applied to the html files.

\item {} 
vegetables-752156\_1280.jpg: background image.

\end{itemize}

\end{enumerate}

\begin{DUlineblock}{0em}
\item[] 
\end{DUlineblock}
\begin{enumerate}
\setcounter{enumi}{6}
\item {} \begin{description}
\item[{uploads folder:}] \leavevmode
Empty folder used to store product images uploaded for products.

\end{description}

\item {} \begin{description}
\item[{docs folder:}] \leavevmode
A folder docs containing the documentation created using Sphinx.

\end{description}

\item {} 
JSON files containing OAuth2.0 parameters:
- client\_secret\_fb.json
- client\_secret\_g.json

\end{enumerate}


\chapter{Python modules}
\label{index:python-modules}

\section{Python Modules}
\label{modules:python-modules}\label{modules::doc}

\subsection{application.py}
\label{application:application-py}\label{application::doc}\label{application:module-application}\index{application (module)}
This module contains the server functionality of Lisa's Grocery Store.
\index{ALLOWED\_EXTENSIONS (in module application)}

\begin{fulllineitems}
\phantomsection\label{application:application.ALLOWED_EXTENSIONS}\pysigline{\code{application.}\bfcode{ALLOWED\_EXTENSIONS}\strong{ = set({[}'gif', `jpeg', `JPG', `jpg', `png'{]})}}
uploaded images can have the extension included in this set.

\end{fulllineitems}

\index{FB\_APP\_ID (in module application)}

\begin{fulllineitems}
\phantomsection\label{application:application.FB_APP_ID}\pysigline{\code{application.}\bfcode{FB\_APP\_ID}\strong{ = u`1610860372517361'}}
str: app id assigned by Facebook and saved in `client\_secret\_fb.json'.

\end{fulllineitems}

\index{FB\_APP\_SECRET (in module application)}

\begin{fulllineitems}
\phantomsection\label{application:application.FB_APP_SECRET}\pysigline{\code{application.}\bfcode{FB\_APP\_SECRET}\strong{ = u'a3585e18fc02dc3f9496c6c7837dcf4f'}}
str: app secret assigned by Facebook and
saved in `client\_secret\_fb.json'.

\end{fulllineitems}

\index{GOOGLE\_CLIENT\_ID (in module application)}

\begin{fulllineitems}
\phantomsection\label{application:application.GOOGLE_CLIENT_ID}\pysigline{\code{application.}\bfcode{GOOGLE\_CLIENT\_ID}\strong{ = u`130832505461-33h5klvebue7cqnh8ojer6drf67e7js3.apps.googleusercontent.com'}}
str: client id assigned by Google and saved in `client\_secret\_g.json'.

\end{fulllineitems}

\index{UPLOAD\_FOLDER (in module application)}

\begin{fulllineitems}
\phantomsection\label{application:application.UPLOAD_FOLDER}\pysigline{\code{application.}\bfcode{UPLOAD\_FOLDER}\strong{ = `./uploads'}}
str: name of the folder where uploaded images can be found.

\end{fulllineitems}

\index{allowed\_file() (in module application)}

\begin{fulllineitems}
\phantomsection\label{application:application.allowed_file}\pysiglinewithargsret{\code{application.}\bfcode{allowed\_file}}{\emph{filename}}{}
Returns True if the filename has an allowed extension.

\end{fulllineitems}

\index{buildProductXML() (in module application)}

\begin{fulllineitems}
\phantomsection\label{application:application.buildProductXML}\pysiglinewithargsret{\code{application.}\bfcode{buildProductXML}}{\emph{product}}{}
Builds up the XML tree for a product.

\end{fulllineitems}

\index{categoriesAtom() (in module application)}

\begin{fulllineitems}
\phantomsection\label{application:application.categoriesAtom}\pysiglinewithargsret{\code{application.}\bfcode{categoriesAtom}}{}{}
Atom API to get info about all categories.

\end{fulllineitems}

\index{categoriesJSON() (in module application)}

\begin{fulllineitems}
\phantomsection\label{application:application.categoriesJSON}\pysiglinewithargsret{\code{application.}\bfcode{categoriesJSON}}{}{}
JSON API to get info about all categories.

\end{fulllineitems}

\index{categoriesXML() (in module application)}

\begin{fulllineitems}
\phantomsection\label{application:application.categoriesXML}\pysiglinewithargsret{\code{application.}\bfcode{categoriesXML}}{}{}
XML API to get info about all categories.

\end{fulllineitems}

\index{categoryAtom() (in module application)}

\begin{fulllineitems}
\phantomsection\label{application:application.categoryAtom}\pysiglinewithargsret{\code{application.}\bfcode{categoryAtom}}{\emph{category\_name}}{}
Atom API to get info about a given category.

\end{fulllineitems}

\index{categoryJSON() (in module application)}

\begin{fulllineitems}
\phantomsection\label{application:application.categoryJSON}\pysiglinewithargsret{\code{application.}\bfcode{categoryJSON}}{\emph{category\_name}}{}
JSON API to get info about a given category.

\end{fulllineitems}

\index{categoryXML() (in module application)}

\begin{fulllineitems}
\phantomsection\label{application:application.categoryXML}\pysiglinewithargsret{\code{application.}\bfcode{categoryXML}}{\emph{category\_name}}{}
XML API to get info about a given category.

\end{fulllineitems}

\index{createLoginOutput() (in module application)}

\begin{fulllineitems}
\phantomsection\label{application:application.createLoginOutput}\pysiglinewithargsret{\code{application.}\bfcode{createLoginOutput}}{}{}
Creates the login output for the login page
after a successful login via Google or Facebook.

\end{fulllineitems}

\index{createUser() (in module application)}

\begin{fulllineitems}
\phantomsection\label{application:application.createUser}\pysiglinewithargsret{\code{application.}\bfcode{createUser}}{\emph{login\_session}}{}
Create a new user in the database.

\end{fulllineitems}

\index{createXMLResponse() (in module application)}

\begin{fulllineitems}
\phantomsection\label{application:application.createXMLResponse}\pysiglinewithargsret{\code{application.}\bfcode{createXMLResponse}}{\emph{tree}}{}
Creates the xml response out of the given tree.

\end{fulllineitems}

\index{csrf\_protect() (in module application)}

\begin{fulllineitems}
\phantomsection\label{application:application.csrf_protect}\pysiglinewithargsret{\code{application.}\bfcode{csrf\_protect}}{}{}
If a POST does not contain a csrf token
or contains a wrong csrf token
a Forbidden is raised.

\end{fulllineitems}

\index{deleteCategory() (in module application)}

\begin{fulllineitems}
\phantomsection\label{application:application.deleteCategory}\pysiglinewithargsret{\code{application.}\bfcode{deleteCategory}}{\emph{category\_name}}{}
Delete the given category and its connected products.

If not logged in yet, the user is redirected to
the login page.
If the request method is POST, the category and all
products of the category are deleted. The images of
the deleted products are also deleted in the upload
folder.
If the request method is GET, the delete category page is
rendered if the user is the owner of the category.

\end{fulllineitems}

\index{deleteProduct() (in module application)}

\begin{fulllineitems}
\phantomsection\label{application:application.deleteProduct}\pysiglinewithargsret{\code{application.}\bfcode{deleteProduct}}{\emph{category\_name}, \emph{product\_name}}{}
Delete a given product of a category.

If not logged in yet, the user is redirected to
the login page.
If the request method is POST, the product is deleted.
The image of the deleted product is also deleted in the
upload folder.
If the request method is GET, the delete product page is
rendered if the user is the owner of the product.

\end{fulllineitems}

\index{editCategory() (in module application)}

\begin{fulllineitems}
\phantomsection\label{application:application.editCategory}\pysiglinewithargsret{\code{application.}\bfcode{editCategory}}{\emph{category\_name}}{}
Edit a given category.

If not logged in yet, the user is redirected to
the login page.
If the request method is POST, the category is updated
with the returned values.
If the request method is GET, the edit category page is
rendered if the user is the owner of the category.

\end{fulllineitems}

\index{editProduct() (in module application)}

\begin{fulllineitems}
\phantomsection\label{application:application.editProduct}\pysiglinewithargsret{\code{application.}\bfcode{editProduct}}{\emph{product\_name}}{}
Edit a given product.

If not logged in yet, the user is redirected to
the login page.
If the request method is POST, the product is updated with
the returned values. The image of the product is also updated
in the upload folder.
If the request method is GET, the edit product page is
rendered if the user is the owner of the product.

\end{fulllineitems}

\index{fbconnect() (in module application)}

\begin{fulllineitems}
\phantomsection\label{application:application.fbconnect}\pysiglinewithargsret{\code{application.}\bfcode{fbconnect}}{}{}
Login the user by Facebook Sign In.

First, the short term token returned by the POST is echanged into
a long term access token. Then get the user info and the picture
and store the user data in the login session for later use.

\end{fulllineitems}

\index{fbdisconnect() (in module application)}

\begin{fulllineitems}
\phantomsection\label{application:application.fbdisconnect}\pysiglinewithargsret{\code{application.}\bfcode{fbdisconnect}}{}{}
Revoke a current fb access token and reset the login\_session.

\end{fulllineitems}

\index{gconnect() (in module application)}

\begin{fulllineitems}
\phantomsection\label{application:application.gconnect}\pysiglinewithargsret{\code{application.}\bfcode{gconnect}}{}{}
Login the user by Google Sign In.

First, obtain the authorization code from the POST, update
the authorization code into a credentials object and check
if the access token within the credentials object is valid.
After having checked that the access token is used for the 
intended user and app, get the user info and store the user
data for later use in the login session.

\end{fulllineitems}

\index{gdisconnect() (in module application)}

\begin{fulllineitems}
\phantomsection\label{application:application.gdisconnect}\pysiglinewithargsret{\code{application.}\bfcode{gdisconnect}}{}{}
Revoke a current google access token and reset the login\_session.

\end{fulllineitems}

\index{generate\_csrf\_token() (in module application)}

\begin{fulllineitems}
\phantomsection\label{application:application.generate_csrf_token}\pysiglinewithargsret{\code{application.}\bfcode{generate\_csrf\_token}}{}{}
Creates a csrf token.

\end{fulllineitems}

\index{getUserID() (in module application)}

\begin{fulllineitems}
\phantomsection\label{application:application.getUserID}\pysiglinewithargsret{\code{application.}\bfcode{getUserID}}{\emph{email}}{}
Returns the user id connected to the given email
or None if the email is not existing in the
database yet.

\end{fulllineitems}

\index{getUserInfo() (in module application)}

\begin{fulllineitems}
\phantomsection\label{application:application.getUserInfo}\pysiglinewithargsret{\code{application.}\bfcode{getUserInfo}}{\emph{user\_id}}{}
Returns a user of the given user\_id.

\end{fulllineitems}

\index{getUsername() (in module application)}

\begin{fulllineitems}
\phantomsection\label{application:application.getUsername}\pysiglinewithargsret{\code{application.}\bfcode{getUsername}}{}{}
Returns the username if already in the
login session, otherwise None.

\end{fulllineitems}

\index{get\_random\_string() (in module application)}

\begin{fulllineitems}
\phantomsection\label{application:application.get_random_string}\pysiglinewithargsret{\code{application.}\bfcode{get\_random\_string}}{}{}
Creates an uppercase random string

\end{fulllineitems}

\index{login() (in module application)}

\begin{fulllineitems}
\phantomsection\label{application:application.login}\pysiglinewithargsret{\code{application.}\bfcode{login}}{}{}
Login function which renders the login page and
starts the login process.

\end{fulllineitems}

\index{logout() (in module application)}

\begin{fulllineitems}
\phantomsection\label{application:application.logout}\pysiglinewithargsret{\code{application.}\bfcode{logout}}{}{}
Logout a user (Google or Facebook).

\end{fulllineitems}

\index{newCategory() (in module application)}

\begin{fulllineitems}
\phantomsection\label{application:application.newCategory}\pysiglinewithargsret{\code{application.}\bfcode{newCategory}}{}{}
Create a new category.

If not logged in yet, the user is redirected to
the login page.
If the request method is POST, a new category is created
in the database with the returned values.
If the request method is GET, the new category page is
rendered.

\end{fulllineitems}

\index{newProduct() (in module application)}

\begin{fulllineitems}
\phantomsection\label{application:application.newProduct}\pysiglinewithargsret{\code{application.}\bfcode{newProduct}}{}{}
Create a new product.

If not logged in yet, the user is redirected to
the login page.
If the request method is POST, a new category is created
in the database with the returned values. The image is
stored in the upload folder.
If the request method is GET, the new product page is
rendered.

\end{fulllineitems}

\index{productAtom() (in module application)}

\begin{fulllineitems}
\phantomsection\label{application:application.productAtom}\pysiglinewithargsret{\code{application.}\bfcode{productAtom}}{\emph{category\_name}, \emph{product\_name}}{}
Atom API to get info about a given product and category.

\end{fulllineitems}

\index{productJSON() (in module application)}

\begin{fulllineitems}
\phantomsection\label{application:application.productJSON}\pysiglinewithargsret{\code{application.}\bfcode{productJSON}}{\emph{category\_name}, \emph{product\_name}}{}
JSON API to get info about a given product and category.

\end{fulllineitems}

\index{productXML() (in module application)}

\begin{fulllineitems}
\phantomsection\label{application:application.productXML}\pysiglinewithargsret{\code{application.}\bfcode{productXML}}{\emph{category\_name}, \emph{product\_name}}{}
XML API to get info about a given product and category.

\end{fulllineitems}

\index{productsAtom() (in module application)}

\begin{fulllineitems}
\phantomsection\label{application:application.productsAtom}\pysiglinewithargsret{\code{application.}\bfcode{productsAtom}}{}{}
Atom API to get info about all products.

\end{fulllineitems}

\index{productsJSON() (in module application)}

\begin{fulllineitems}
\phantomsection\label{application:application.productsJSON}\pysiglinewithargsret{\code{application.}\bfcode{productsJSON}}{}{}
JSON API to get info about all products.

\end{fulllineitems}

\index{productsXML() (in module application)}

\begin{fulllineitems}
\phantomsection\label{application:application.productsXML}\pysiglinewithargsret{\code{application.}\bfcode{productsXML}}{}{}
XML API to get info about all products.

\end{fulllineitems}

\index{resetUserSession() (in module application)}

\begin{fulllineitems}
\phantomsection\label{application:application.resetUserSession}\pysiglinewithargsret{\code{application.}\bfcode{resetUserSession}}{\emph{result}}{}
resets the user data of the login session.

\end{fulllineitems}

\index{showCategory() (in module application)}

\begin{fulllineitems}
\phantomsection\label{application:application.showCategory}\pysiglinewithargsret{\code{application.}\bfcode{showCategory}}{\emph{category\_name}}{}
Show the category and the products of the category.

\end{fulllineitems}

\index{showGroceryStore() (in module application)}

\begin{fulllineitems}
\phantomsection\label{application:application.showGroceryStore}\pysiglinewithargsret{\code{application.}\bfcode{showGroceryStore}}{}{}
Show the grocerystore with all categories
and all products existing in the database.

\end{fulllineitems}

\index{showProduct() (in module application)}

\begin{fulllineitems}
\phantomsection\label{application:application.showProduct}\pysiglinewithargsret{\code{application.}\bfcode{showProduct}}{\emph{category\_name}, \emph{product\_name}}{}
Show one given product of a given category.

\end{fulllineitems}

\index{storeUserData() (in module application)}

\begin{fulllineitems}
\phantomsection\label{application:application.storeUserData}\pysiglinewithargsret{\code{application.}\bfcode{storeUserData}}{\emph{username}, \emph{email}, \emph{picture}, \emph{ext\_user\_id}, \emph{google\_credentials=None}, \emph{fb\_access\_token=None}}{}
Stores the user data
given by Google or Facebook
in the login session.

\end{fulllineitems}

\index{uploads() (in module application)}

\begin{fulllineitems}
\phantomsection\label{application:application.uploads}\pysiglinewithargsret{\code{application.}\bfcode{uploads}}{\emph{filename}}{}
Locates the given file in the upload directory and shows
it in the browser.

\end{fulllineitems}



\subsection{database\_setup.py}
\label{database_setup:database-setup-py}\label{database_setup::doc}\label{database_setup:module-database_setup}\index{database\_setup (module)}
Database setup for Lisa's Grocery Store.
The database consists of three tables representing
three objects:
\begin{itemize}
\item {} 
User:
A user is created after login and can
create, update or delete categories and products.

\item {} 
Category:
A category can be created, updated and deleted.
It is always owned by one user.

\item {} 
Product:
A product can be created, updated and deleted.
It is always owned by one user and assigned
to one category.

\end{itemize}
\index{Category (class in database\_setup)}

\begin{fulllineitems}
\phantomsection\label{database_setup:database_setup.Category}\pysiglinewithargsret{\strong{class }\code{database\_setup.}\bfcode{Category}}{\emph{**kwargs}}{}
Bases: \code{sqlalchemy.ext.declarative.api.Base}

Class representing a Category.
\index{description (database\_setup.Category attribute)}

\begin{fulllineitems}
\phantomsection\label{database_setup:database_setup.Category.description}\pysigline{\bfcode{description}}
description (Column): a description of the category.

\end{fulllineitems}

\index{name (database\_setup.Category attribute)}

\begin{fulllineitems}
\phantomsection\label{database_setup:database_setup.Category.name}\pysigline{\bfcode{name}}
name (Column): the category name (primary key).

\end{fulllineitems}

\index{serialize (database\_setup.Category attribute)}

\begin{fulllineitems}
\phantomsection\label{database_setup:database_setup.Category.serialize}\pysigline{\bfcode{serialize}}
Return object data in easily serializeable format.

\end{fulllineitems}

\index{user (database\_setup.Category attribute)}

\begin{fulllineitems}
\phantomsection\label{database_setup:database_setup.Category.user}\pysigline{\bfcode{user}}
relation to the user who is the owner.

\end{fulllineitems}

\index{user\_id (database\_setup.Category attribute)}

\begin{fulllineitems}
\phantomsection\label{database_setup:database_setup.Category.user_id}\pysigline{\bfcode{user\_id}}
user\_id (Column): reference to the owner of the category.

\end{fulllineitems}


\end{fulllineitems}

\index{Product (class in database\_setup)}

\begin{fulllineitems}
\phantomsection\label{database_setup:database_setup.Product}\pysiglinewithargsret{\strong{class }\code{database\_setup.}\bfcode{Product}}{\emph{**kwargs}}{}
Bases: \code{sqlalchemy.ext.declarative.api.Base}

Class representing an product.
\index{category (database\_setup.Product attribute)}

\begin{fulllineitems}
\phantomsection\label{database_setup:database_setup.Product.category}\pysigline{\bfcode{category}}
\end{fulllineitems}

\index{category\_name (database\_setup.Product attribute)}

\begin{fulllineitems}
\phantomsection\label{database_setup:database_setup.Product.category_name}\pysigline{\bfcode{category\_name}}
category\_name (Column): the category the product belongs to.

\end{fulllineitems}

\index{description (database\_setup.Product attribute)}

\begin{fulllineitems}
\phantomsection\label{database_setup:database_setup.Product.description}\pysigline{\bfcode{description}}
description (Column): a description of the product.

\end{fulllineitems}

\index{image\_file\_name (database\_setup.Product attribute)}

\begin{fulllineitems}
\phantomsection\label{database_setup:database_setup.Product.image_file_name}\pysigline{\bfcode{image\_file\_name}}
image\_file\_name (str): file name of an image of the product.
The image is stored in the upload folder.

\end{fulllineitems}

\index{name (database\_setup.Product attribute)}

\begin{fulllineitems}
\phantomsection\label{database_setup:database_setup.Product.name}\pysigline{\bfcode{name}}
name (Column): the product name, primary key.

\end{fulllineitems}

\index{price (database\_setup.Product attribute)}

\begin{fulllineitems}
\phantomsection\label{database_setup:database_setup.Product.price}\pysigline{\bfcode{price}}
price (Column): the price of the product.

\end{fulllineitems}

\index{serialize (database\_setup.Product attribute)}

\begin{fulllineitems}
\phantomsection\label{database_setup:database_setup.Product.serialize}\pysigline{\bfcode{serialize}}
Return object data in easily serializeable format.

\end{fulllineitems}

\index{user (database\_setup.Product attribute)}

\begin{fulllineitems}
\phantomsection\label{database_setup:database_setup.Product.user}\pysigline{\bfcode{user}}
relation to the user who is the owner.

\end{fulllineitems}

\index{user\_id (database\_setup.Product attribute)}

\begin{fulllineitems}
\phantomsection\label{database_setup:database_setup.Product.user_id}\pysigline{\bfcode{user\_id}}
user\_id (Column): the id of the owner of the product.

\end{fulllineitems}


\end{fulllineitems}

\index{User (class in database\_setup)}

\begin{fulllineitems}
\phantomsection\label{database_setup:database_setup.User}\pysiglinewithargsret{\strong{class }\code{database\_setup.}\bfcode{User}}{\emph{**kwargs}}{}
Bases: \code{sqlalchemy.ext.declarative.api.Base}

Class representing a user.
\index{email (database\_setup.User attribute)}

\begin{fulllineitems}
\phantomsection\label{database_setup:database_setup.User.email}\pysigline{\bfcode{email}}
email (Column): the email of the user.

\end{fulllineitems}

\index{id (database\_setup.User attribute)}

\begin{fulllineitems}
\phantomsection\label{database_setup:database_setup.User.id}\pysigline{\bfcode{id}}
id (Column): the internal user id used as primary key.

\end{fulllineitems}

\index{name (database\_setup.User attribute)}

\begin{fulllineitems}
\phantomsection\label{database_setup:database_setup.User.name}\pysigline{\bfcode{name}}
name (Column): the user name.

\end{fulllineitems}

\index{picture (database\_setup.User attribute)}

\begin{fulllineitems}
\phantomsection\label{database_setup:database_setup.User.picture}\pysigline{\bfcode{picture}}
picture (Column): link to a picture of the user.

\end{fulllineitems}

\index{serialize (database\_setup.User attribute)}

\begin{fulllineitems}
\phantomsection\label{database_setup:database_setup.User.serialize}\pysigline{\bfcode{serialize}}
Return object data in easily serializeable format.

\end{fulllineitems}


\end{fulllineitems}



\subsection{load\_grocery\_store.py}
\label{load_grocery_store:module-load_grocery_store}\label{load_grocery_store:load-grocery-store-py}\label{load_grocery_store::doc}\index{load\_grocery\_store (module)}
Populates the database of Lisa's Grocery Store with a few
Users, Categories and Products.
\index{DBSession (in module load\_grocery\_store)}

\begin{fulllineitems}
\phantomsection\label{load_grocery_store:load_grocery_store.DBSession}\pysigline{\code{load\_grocery\_store.}\bfcode{DBSession}\strong{ = sessionmaker(class\_='Session',autoflush=True, bind=Engine(sqlite:///grocery\_store.db), autocommit=False, expire\_on\_commit=True)}}
A DBSession() instance establishes all conversations with the database
and represents a ``staging zone'' for all the objects loaded into the
database session object. Any change made against the objects in the
session won't be persisted into the database until you call
session.commit(). If you're not happy about the changes, you can
revert all of them back to the last commit by calling
session.rollback().

\end{fulllineitems}

\index{engine (in module load\_grocery\_store)}

\begin{fulllineitems}
\phantomsection\label{load_grocery_store:load_grocery_store.engine}\pysigline{\code{load\_grocery\_store.}\bfcode{engine}\strong{ = Engine(sqlite:///grocery\_store.db)}}
Bind the engine to the metadata of the Base class so that the
declaratives can be accessed through a DBSession instance.

\end{fulllineitems}



\chapter{API Endpoints}
\label{index:api-endpoints}
JSON:
\begin{itemize}
\item {} 
\href{http://localhost:8000/grocerystore/categories/JSON}{http://localhost:8000/grocerystore/categories/JSON}

\item {} 
\href{http://localhost:8000/grocerystore/products/JSON}{http://localhost:8000/grocerystore/products/JSON}

\item {} 
\href{http://localhost:8000/grocerystore}{http://localhost:8000/grocerystore}/\textless{}category\_name\textgreater{}/products/JSON

\item {} 
\href{http://localhost:8000/grocerystore}{http://localhost:8000/grocerystore}/\textless{}category\_name\textgreater{}/\textless{}product\_name\textgreater{}/JSON

\end{itemize}

XML:
\begin{itemize}
\item {} 
\href{http://localhost:8000/grocerystore/categories/XML}{http://localhost:8000/grocerystore/categories/XML}

\item {} 
\href{http://localhost:8000/grocerystore/products/XML}{http://localhost:8000/grocerystore/products/XML}

\item {} 
\href{http://localhost:8000/grocerystore}{http://localhost:8000/grocerystore}/\textless{}category\_name\textgreater{}/products/XML

\item {} 
\href{http://localhost:8000/grocerystore}{http://localhost:8000/grocerystore}/\textless{}category\_name\textgreater{}/\textless{}product\_name\textgreater{}/XML

\end{itemize}

Atom:
\begin{itemize}
\item {} 
\href{http://localhost:8000/grocerystore/categories/Atom}{http://localhost:8000/grocerystore/categories/Atom}

\item {} 
\href{http://localhost:8000/grocerystore/products/Atom}{http://localhost:8000/grocerystore/products/Atom}

\item {} 
\href{http://localhost:8000/grocerystore}{http://localhost:8000/grocerystore}/\textless{}category\_name\textgreater{}/products/Atom

\item {} 
\href{http://localhost:8000/grocerystore}{http://localhost:8000/grocerystore}/\textless{}category\_name\textgreater{}/\textless{}product\_name\textgreater{}/Atom

\end{itemize}


\chapter{Indices and tables}
\label{index:indices-and-tables}\begin{itemize}
\item {} 
\DUspan{xref,std,std-ref}{genindex}

\item {} 
\DUspan{xref,std,std-ref}{modindex}

\item {} 
\DUspan{xref,std,std-ref}{search}

\end{itemize}


\chapter{Contact}
\label{index:contact}
\href{mailto:lisa.kugler@googlemail.com}{lisa.kugler@googlemail.com}


\renewcommand{\indexname}{Python Module Index}
\begin{theindex}
\def\bigletter#1{{\Large\sffamily#1}\nopagebreak\vspace{1mm}}
\bigletter{a}
\item {\texttt{application}}, \pageref{application:module-application}
\indexspace
\bigletter{d}
\item {\texttt{database\_setup}}, \pageref{database_setup:module-database_setup}
\indexspace
\bigletter{l}
\item {\texttt{load\_grocery\_store}}, \pageref{load_grocery_store:module-load_grocery_store}
\end{theindex}

\renewcommand{\indexname}{Index}
\printindex
\end{document}
